%%
%% This is file `sample-sigplan.tex',
%% generated with the docstrip utility.
%%
%% The original source files were:
%%
%% samples.dtx  (with options: `sigplan')
%% 
%% IMPORTANT NOTICE:
%% 
%% For the copyright see the source file.
%% 
%% Any modified versions of this file must be renamed
%% with new filenames distinct from sample-sbaseigplan.tex.
%% 
%% For distribution of the original source see the terms
%% for copying and modification in the file samples.dtx.
%% 
%% This generated file may be distributed as long as the
%% original source files, as listed above, are part of the
%% same distribution. (The sources need not necessarily be
%% in the same archive or directory.)
%%
%% The first command in your LaTeX source must be the \documentclass command.
\documentclass[sigplan,screen]{acmart}

%%
%% \BibTeX command to typeset BibTeX logo in the docs
\AtBeginDocument{%
  \providecommand\BibTeX{{%
    \normalfont B\kern-0.5em{\scshape i\kern-0.25em b}\kern-0.8em\TeX}}}

% Remove ACM reference format thingy
\settopmatter{printacmref=false}

%% Rights management information.  This information is sent to you
%% when you complete the rights form.  These commands have SAMPLE
%% values in them; it is your responsibility as an author to replace
%% the commands and values with those provided to you when you
%% complete the rights form.
% \setcopyright{acmcopyright}
% \copyrightyear{2018}
% \acmYear{2018}
% \acmDOI{10.1145/1122445.1122456}

%% These commands are for a PROCEEDINGS abstract or paper.
% \acmConference[EuroSys '20]{EuroSys '20}{April 27--30, 2020}{Heraklion, Greece}
% \acmISBN{978-1-4503-XXXX-X/18/06}


%%
%% Submission ID.
%% Use this when submitting an article to a sponsored event. You'll
%% receive a unique submission ID from the organizers
%% of the event, and this ID should be used as the parameter to this command.
%%\acmSubmissionID{123-A56-BU3}

%%
%% The majority of ACM publications use numbered citations and
%% references.  The command \citestyle{authoryear} switches to the
%% "author year" style.
%%
%% If you are preparing content for an event
%% sponsored by ACM SIGGRAPH, you must use the "author year" style of
%% citations and references.
%% Uncommenting
%% the next command will enable that style.
%%\citestyle{acmauthoryear}

%% remove copyright box at the bottom
\setcopyright{none}

%%
%% end of the preamble, start of the body of the document source.
\begin{document}

%%
%% The "title" command has an optional parameter,
%% allowing the author to define a "short title" to be used in page headers.
\title{Finesse}

%%
%% The "author" command and its associated commands are used to define
%% the authors and their affiliations.
%% Of note is the shared affiliation of the first two authors, and the
%% "authornote" and "authornotemark" commands
%% used to denote shared contribution to the research.
\author{Tony Mason}
\affiliation{%
  \institution{University of British Columbia}
}
\email{fsgeek@cs.ubc.ca}


\author{Matheus Stolet}
\affiliation{%
  \institution{University of British Columbia}
}
\email{stolet@cs.ubc.ca}

%%
%% The abstract is a short summary of the work to be presented in the
%% article.
\begin{abstract}
  In-kernel file systems are challenging to develop. Program-
  mers involved in kernel development have to deal with bugs
  that cause fatal system crashes. Userspace file systems mitigate
  this problem by allowing developers to anchor themselves in
  a forgiving user-space environment. Unfortunately, popular
  solutions to file systems in user-space, such as FUSE \cite{fusegithub}, add
  significant overhead to operations, making them unpalatable
  to many applications. Finesse solves this problem
  by adding a client library and a thin software layer that sits
  on top of the FUSE library. The Finesse layer allows the user
  to bypass the kernel for selected operations, so that applica-
  tions communicate directly with the FUSE file system. New
  opperations can be added to the application library, enabling
  applications to extend the file system by creating their own set
  of custom operations. Ultimately, Finesse is a kernel
  bypass system meant to improve the performance of user-space file
  systems, without increasing development complexity.

  Finesse achieves better performance by modifying FUSE
  to serve multiple message sources. This is done by implementing
  a message passing interface that allows applications to make direct
  calls to the FUSE file system, effectively avoiding the convoluted kernel path
  taken by FUSE. If an operation is not implemented
  in the Finesse layer, it falls back to regular FUSE behaviour,
  allowing the file system to serve operations coming from Finesse
  and the FUSE kernel driver. 
  
  There are two major pieces to the Finesse
  system: the Finesse application library and the Finesse FUSE extension. 
  The Finesse application library is a user-facing library that allows the explicit
  invocation of operations implemented in the Finesse library and the implicit invocation
  of operations through LD\_PRELOAD, that loads the Finesse library before other shared libraries such as
  \textit{libc}. The Finesse FUSE extension lies between the FUSE file system and the FUSE library.
  It listens to incoming file system operations from the Finesse library and redirects them to the
  FUSE file system.

  Finesse is still in development, and we are working at porting a bigger number of file system
  operations to use the Finesse library instead of defaulting to regular FUSE behaviour. Preliminary
  evaluations have showed that the implementation of \textit{unlink} using the Finesse library led to
  considerable performance advantages. For example, when deleting 4 million preallocated 4KB files over
  many directories, regular FUSE had a 14.94\% performance overhead, while FUSE with Finesse led to a 
  29.21\% performance gain.


\end{abstract}

%%
%% Keywords. The author(s) should pick words that accurately describe
%% the work being presented. Separate the keywords with commas.
\keywords{file systems, operating systems, user-space, kernel bypass}

%%
%% This command processes the author and affiliation and title
%% information and builds the first part of the formatted document.
\maketitle

% \section{Introduction}
% ACM's consolidated article template, introduced in 2017, provides a
% consistent \LaTeX\ style for use across ACM publications, and
% incorporates accessibility and metadata-extraction functionality
% necessary for future Digital Library endeavors. Numerous ACM and
% SIG-specific \LaTeX\ templates have been examined, and their unique
% features incorporated into this single new template.

% If you are new to publishing with ACM, this document is a valuable
% guide to the process of preparing your work for publication. If you
% have published with ACM before, this document provides insight and
% instruction into more recent changes to the article template.

% The ``\verb|acmart|'' document class can be used to prepare articles
% for any ACM publication --- conference or journal, and for any stage
% of publication, from review to final ``camera-ready'' copy, to the
% author's own version, with {\itshape very} few changes to the source.

%%
%% The next two lines define the bibliography style to be used, and
%% the bibliography file.
\bibliographystyle{ACM-Reference-Format}
\bibliography{finesse_poster}

\end{document}
\endinput
%%
%% End of file `sample-sigplan.tex'.
