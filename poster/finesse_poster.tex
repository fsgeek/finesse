
%!TEX program = lualatex
\documentclass[sigplan,10pt,letter,plain]{acmart}
\settopmatter{printacmref=false} % Removes citation information below abstract
\renewcommand\footnotetextcopyrightpermission[1]{}
\pagestyle{plain}

%% Rights management information.  This information is sent to you
%% when you complete the rights form.  These commands have SAMPLE
%% values in them; it is your responsibility as an author to replace
%% the commands and values with those provided to you when you
%% complete the rights form.
% \setcopyright{acmcopyright}
% \copyrightyear{2018}
% \acmYear{2018}
% \acmDOI{10.1145/1122445.1122456}

%% These commands are for a PROCEEDINGS abstract or paper.
%\acmConference[EuroSys '20]{EuroSys '20}{April 27--30, 2020}{Heraklion, Greece}
% \acmISBN{978-1-4503-XXXX-X/18/06}


%%
%% Submission ID.
%% Use this when submitting an article to a sponsored event. You'll
%% receive a unique submission ID from the organizers
%% of the event, and this ID should be used as the parameter to this command.
%%\acmSubmissionID{123-A56-BU3}

%%
%% The majority of ACM publications use numbered citations and
%% references.  The command \citestyle{authoryear} switches to the
%% "author year" style.
%%
%% If you are preparing content for an event
%% sponsored by ACM SIGGRAPH, you must use the "author year" style of
%% citations and references.
%% Uncommenting
%% the next command will enable that style.
%%\citestyle{acmauthoryear}

%% remove copyright box at the bottom
%\setcopyright{none}

%%
%% end of the preamble, start of the body of the document source.
\begin{document}

%%
%% The "title" command has an optional parameter,
%% allowing the author to define a "short title" to be used in page headers.
\title{Finesse: Kernel Bypass for File Systems}

%%
%% The "author" command and its associated commands are used to define
%% the authors and their affiliations.
%% Of note is the shared affiliation of the first two authors, and the
%% "authornote" and "authornotemark" commands
%% used to denote shared contribution to the research.

\author{Matheus Stolet}
\authornote{Student, Presenter}
\affiliation{%
  \institution{University of British Columbia}
}
\email{stolet@cs.ubc.ca}
\author{Tony Mason}
\authornote{Student}
\affiliation{%
  \institution{University of British Columbia}
}
\email{fsgeek@cs.ubc.ca}

\renewcommand{\shortauthors}{M. Stolet et al.}

%%
%% The abstract is a short summary of the work to be presented in the
%% article.
\begin{abstract}
  Finesse implements a kernel bypass technique for enhancing the performance of the popular cross-platform FUSE file system framework, without requiring intrusive changes to existing applications \textit{or} user-mode file systems.  Further, Finesse provides a more extensible model that permits enhancement of applications by shifting expensive meta-data intensive operations into the FUSE file system.  Thus, Finesse offers a compelling improvement for FUSE file system developers: existing applications benefit from improved performance without program changes and without requiring file system source code changes, as well as permitting application developers to exploit new functionality that is exposed by the Finesse-enhanced file system, as well as functionality extensions implemented by file systems developers.
\end{abstract}

%%
%% Keywords. The author(s) should pick words that accurately describe
%% the work being presented. Separate the keywords with commas.
% \keywords{file systems, operating systems, user-space, kernel bypass}

%%
%% This command processes the author and affiliation and title
%% information and builds the first part of the formatted document.
\maketitle

%\balance - can't use that with the footnotes; doesn't seem to handle it quite right.  Maybe once the page is full?

\section{Introduction}

Kernel programming is notorious for being a challenging development environment.  Despite this complexity, production use file systems are frequently implemented for in-kernel execution because they offer the best performance.  Programming in user space is more forgiving and has a broader range of well-supported languages, libraries, and options that are not available in the kernel environment. The cost of using userspace file systems development tools, such as FUSE~\cite{fusegithub}, is typically performance.  Recent work has pointed this out and looked at various ways of improving performance~\cite{vangoor2017fuse,10.1145/3341301.3359637}.

Prior work has explored various ways of improving performance, including interception libraries~\cite{wright2012ldplfs} and kernel mode extensions for optimizing data copy~\cite{234870}.  The idea of implementing a hybrid combination of both kernel bypass and fallback kernel support has not been explored.

\section{Description}

Finesse explores this hybrid environment by modifying the behavior of the existing system libraries, which enables enhancing existing applications without requiring any program changes, and by modifying the FUSE file system library, which enables supporting existing FUSE file systems with a simple recompilation.

Finesse uses a simple message passing model for implementing a kernel bypass between the applications and user mode FUSE file systems.  Existing applications can eschew using Finesse by using the original system libraries without any loss of functionality.  Currently, we achieve replacement of existing system libraries on Linux by using the LD\_PRELOAD mechanism, which ensures the Finesse application library is invoked first.  When Finesse determines that the call is directed to a mounted FUSE file system, it converts that call to a message, which is passed directly to the FUSE file system.

We have observed up to a 61\% performance improvement with Finesse with the \textit{unlink} system call, versus using the standard FUSE interface. In this test, we added \textit{unlink} support to the FUSE pass-through file system, preallocated four million 4KB files in multiple directories, and measured the time to delete them.  Finesse was 29\% \textit{faster} than the native file system, and 61\% faster than FUSE alone.  Finesse is in active development and we continue to increase the number of file system operations supported by Finesse.

\section{Conclusion}

Finesse combines a kernel bypass message passing architecture with a fallback traditional file systems support layer.  Preliminary results for this approach have achieved up to 62\% better performance on one operations.  We anticipate further positive results as we continue expanding our work. In future, we will also explore adding file systems API enhancements that will yield both better performance as well as novel functionality.


% \section{Introduction}
% ACM's consolidated article template, introduced in 2017, provides a
% consistent \LaTeX\ style for use across ACM publications, and
% incorporates accessibility and metadata-extraction functionality
% necessary for future Digital Library endeavors. Numerous ACM and
% SIG-specific \LaTeX\ templates have been examined, and their unique
% features incorporated into this single new template.

% If you are new to publishing with ACM, this document is a valuable
% guide to the process of preparing your work for publication. If you
% have published with ACM before, this document provides insight and
% instruction into more recent changes to the article template.

% The ``\verb|acmart|'' document class can be used to prepare articles
% for any ACM publication --- conference or journal, and for any stage
% of publication, from review to final ``camera-ready'' copy, to the
% author's own version, with {\itshape very} few changes to the source.

%%
%% The next two lines define the bibliography style to be used, and
%% the bibliography file.
\bibliographystyle{ACM-Reference-Format}
\bibliography{finesse_poster}

\end{document}
\endinput
%%
%% End of file `sample-sigplan.tex'.
