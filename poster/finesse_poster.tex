
%!TEX program = lualatex
\documentclass[sigplan,10pt,letter,plain]{acmart}
\settopmatter{printacmref=false} % Removes citation information below abstract
\renewcommand\footnotetextcopyrightpermission[1]{}
\pagestyle{plain}

%% Rights management information.  This information is sent to you
%% when you complete the rights form.  These commands have SAMPLE
%% values in them; it is your responsibility as an author to replace
%% the commands and values with those provided to you when you
%% complete the rights form.
% \setcopyright{acmcopyright}
% \copyrightyear{2018}
% \acmYear{2018}
% \acmDOI{10.1145/1122445.1122456}

%% These commands are for a PROCEEDINGS abstract or paper.
%\acmConference[EuroSys '20]{EuroSys '20}{April 27--30, 2020}{Heraklion, Greece}
% \acmISBN{978-1-4503-XXXX-X/18/06}


%%
%% Submission ID.
%% Use this when submitting an article to a sponsored event. You'll
%% receive a unique submission ID from the organizers
%% of the event, and this ID should be used as the parameter to this command.
%%\acmSubmissionID{123-A56-BU3}

%%
%% The majority of ACM publications use numbered citations and
%% references.  The command \citestyle{authoryear} switches to the
%% "author year" style.
%%
%% If you are preparing content for an event
%% sponsored by ACM SIGGRAPH, you must use the "author year" style of
%% citations and references.
%% Uncommenting
%% the next command will enable that style.
%%\citestyle{acmauthoryear}

%% remove copyright box at the bottom
%\setcopyright{none}

%%
%% end of the preamble, start of the body of the document source.
\begin{document}

%%
%% The "title" command has an optional parameter,
%% allowing the author to define a "short title" to be used in page headers.
\title{Finesse: Kernel Bypass for File Systems}

%%
%% The "author" command and its associated commands are used to define
%% the authors and their affiliations.
%% Of note is the shared affiliation of the first two authors, and the
%% "authornote" and "authornotemark" commands
%% used to denote shared contribution to the research.

\author{Matheus Stolet}
\authornote{Student, Presenter}
\affiliation{%
  \institution{University of British Columbia}
}
\email{stolet@cs.ubc.ca}
\author{Tony Mason}
\authornote{Student}
\affiliation{%
  \institution{University of British Columbia}
}
\email{fsgeek@cs.ubc.ca}

\renewcommand{\shortauthors}{M. Stolet et al.}

%%
%% The abstract is a short summary of the work to be presented in the
%% article.
\begin{abstract}
  Developing kernel file systems presents challenges due to the complexity of the environment.  The inflexibility of this environment discourages exploration of alternatives; most file systems research is focused on improving the storage management aspects, rather than the application facing interfaces.  While userspace file systems mitigate the development complexity, they do so with a significant performance penalty and little flexibility for exploring novel access models.  Finesse addresses these two concerns by adding both a client side library and a FUSE file system library.  Finesse implements a kernel bypass mechanism for key operations and enables making new interfaces available to applications.  Improving performance while balancing compatibility against flexibility to explore alternative interfaces is compelling for userspace file systems development.
\end{abstract}

%%
%% Keywords. The author(s) should pick words that accurately describe
%% the work being presented. Separate the keywords with commas.
% \keywords{file systems, operating systems, user-space, kernel bypass}

%%
%% This command processes the author and affiliation and title
%% information and builds the first part of the formatted document.
\maketitle

%\balance - can't use that with the footnotes; doesn't seem to handle it quite right.  Maybe once the page is full?

\section{Introduction}

Kernel programming is notorious for being a challenging development environment.  Despite this complexity, production use file systems are frequently implemented for in-kernel execution because they offer the best performance.  Programming in user space is more forgiving and has a broader range of well-supported languages, libraries, and options that are not available in the kernel environment. The cost of using userspace file systems development tools, such as FUSE~\cite{fusegithub}, is typically performance.  Recent work has pointed this out and looked at various ways of improving performance~\cite{vangoor2017fuse,10.1145/3341301.3359637}.

Prior work has explored various ways of improving performanc, including interception libraries~\cite{wright2012ldplfs} and kernel mode extensions for optimizing data copy~\cite{234870}.  The idea of implementing a hybrid combination of both kernel bypass and fallback kernel support has not been explored.

\section{Model}

Finesse explores such a hybrid environment by introducing a client side shared library that intercepts a subset of calls and provides an alternative implementation, as well as an enhanced FUSE library model.  This combination permits Finesse to provide an efficient kernel bypass mechanism.  This is achieved using an efficient message passing interface between the client application and the FUSE file system library.  Operations that are not implemented by the Finesse library fall back to system calls and standard FUSE behavior.

The Finesse application library is a user-facing library that allows the explicit invocation of operations implemented in the Finesse library and the implicit invocation of operations through LD\_PRELOAD, that loads the Finesse library before other shared libraries such as \textit{libc}. The Finesse FUSE extension lies between the FUSE file system and the FUSE library.   It listens to incoming file system operations from the Finesse library and redirects them to the FUSE file system.

\section{Evaluation}

Finesse is still in development, and we are working at porting a bigger number of file system operations to use the Finesse library instead of defaulting to regular FUSE behavior. Preliminary evaluations have showed that the implementation of \textit{unlink} using the Finesse library led to considerable performance improvements for a number of operations. For example, when deleting 4 million preallocated 4KB files over many directories, the existing FUSE library we tested showed a 15\% decrease in performance.  The same test done with the Finesse+FUSE library yielded a 29\% performance gain. 
%\textbf{How did this compare with the native file system?}


\section{Conclusion}

Finesse combines a kernel bypass message passing architecture with a fallback traditional file systems support layer.  Preliminary results for this approach have achieved up to 44\% better performance on specific operations, and we anticipate further positive results as we continue expanding our work.
In future, we will explore adding file systems API enhancements that will yield both better performance as well as novel functionality.


% \section{Introduction}
% ACM's consolidated article template, introduced in 2017, provides a
% consistent \LaTeX\ style for use across ACM publications, and
% incorporates accessibility and metadata-extraction functionality
% necessary for future Digital Library endeavors. Numerous ACM and
% SIG-specific \LaTeX\ templates have been examined, and their unique
% features incorporated into this single new template.

% If you are new to publishing with ACM, this document is a valuable
% guide to the process of preparing your work for publication. If you
% have published with ACM before, this document provides insight and
% instruction into more recent changes to the article template.

% The ``\verb|acmart|'' document class can be used to prepare articles
% for any ACM publication --- conference or journal, and for any stage
% of publication, from review to final ``camera-ready'' copy, to the
% author's own version, with {\itshape very} few changes to the source.

%%
%% The next two lines define the bibliography style to be used, and
%% the bibliography file.
\bibliographystyle{ACM-Reference-Format}
\bibliography{finesse_poster}

\end{document}
\endinput
%%
%% End of file `sample-sigplan.tex'.
